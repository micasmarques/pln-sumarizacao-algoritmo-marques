In this work, we address the context of automatic text summarization, an area in constant development with significant challenges to overcome. The main problem is the comparison of the performance of six summarization algorithms, identifying the most suitable one for generating automatic summaries of documents in this language. The relevance of this study lies in the need to provide efficient automatic summarization tools to help users extract relevant information from Portuguese texts quickly and effectively.
The contributions of this work include the application and analysis of six automatic summarization algorithms, namely: Luhn's Algorithm, GistSumm, ChatGPT, Integer Linear Programming Algorithm, Bayesian Regression Algorithm, and Marques' Algorithm. The methodology involved applying the algorithms to a text about COVID-19 and its impacts on the pandemic, and evaluating the results based on summarization quality metrics, as well as conducting a user study to verify the usefulness and relevance of the generated summaries.
Marques' Algorithm outperformed the other algorithms in terms of precision, coherence, cohesion, and processing time, being identified as a promising choice for generating automatic summaries of documents in Portuguese. However, we emphasize that the area still faces challenges to overcome, such as adapting to different types of texts and domains of knowledge, rigorous and consistent evaluation of generated summaries, and generating personalized summaries according to users' needs.
In conclusion, this work contributed to the advancement of the automatic text summarization area by comparing the performance of six summarization algorithms and identifying Marques' Algorithm as an effective option. 
% Separe as Keywords por ponto

\keywords{Automatic Summarization. Integer Linear Programming Algorithm. GistSumm. Luhn's Algorithm. ChatGPT. Bayesian Regression Algorithm. Machine Learning. Natural Language Processing}