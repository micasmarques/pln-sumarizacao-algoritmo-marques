Neste trabalho, aborda-se o contexto da sumarização automática de texto, uma área em constante desenvolvimento e com desafios significativos a serem superados. O problema principal é a comparação do desempenho de seis algoritmos de sumarização, identificando o mais adequado para gerar resumos automáticos de documentos nesse idioma. A relevância do estudo reside na necessidade de disponibilizar ferramentas eficientes de sumarização automática para auxiliar os usuários a extrair informações relevantes de textos em português de forma rápida e eficaz.
As contribuições do trabalho incluem a aplicação e análise de seis algoritmos de sumarização automática, a saber: Algoritmo de Luhn, \textit{GistSumm}, \textit{ChatGPT}, Algoritmo de Programação Linear Inteira, Algoritmo de Regressão Bayesiana e Algoritmo de Marques. A metodologia envolveu a aplicação dos algoritmos em um texto sobre a COVID-19 e seus impactos na pandemia, e a avaliação dos resultados com base em métricas de qualidade de sumarização, bem como a realização de um estudo com usuários para verificar a utilidade e relevância dos resumos gerados.
O algoritmo de Marques apresentou desempenho superior em relação aos demais nas métricas de precisão, coerência, coesão e tempo de processamento, sendo identificado como uma escolha promissora para a geração de resumos automáticos de documentos em português. No entanto, ressalta-se que a área ainda enfrenta desafios a serem superados, como a adaptação a diferentes tipos de textos e domínios de conhecimento, a avaliação rigorosa e consistente dos resumos gerados e a geração de resumos personalizados de acordo com as necessidades dos usuários.
Em conclusão, este trabalho contribuiu para o avanço da área de sumarização automática de texto, ao comparar o desempenho de seis algoritmos de sumarização e identificar o algoritmo de Marques como uma opção eficaz. 
% Separe as Keywords por ponto

\keywords{Sumarização Automática. Algoritmo de Programação Linear Inteira. GistSumm. Algoritmo de Luhn. ChatGPT. Algoritmo de Regressão Bayesiana. Aprendizado de Máquina. Processamento de Linguagem Natural}