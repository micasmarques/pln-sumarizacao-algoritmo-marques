Em primeiro lugar, quero expressar minha profunda gratidão aos meus pais, José Expedito da Silva e 
Isabel Ana Rodrigues da Silva, por todo o amor, apoio e encorajamento que me proporcionaram ao longo 
de toda a minha vida e, especialmente, durante o desenvolvimento desta monografia.

Agradeço ao meu irmão, Misael Rodrigues Silva, pelo companheirismo e inspiração, que me motivaram a 
continuar meus estudos e a buscar meus objetivos.

Gostaria de estender meus sinceros agradecimentos ao meu primo Rubens de Oliveira Rodrigues e sua 
família, que generosamente me ofereceram sua casa, apoio e conforto, permitindo-me focar nos meus 
estudos enquanto estava longe da minha família. Serei eternamente grato pela sua hospitalidade e 
carinho.

Também quero homenagear meus avós, que já se foram, mas que sempre foram uma fonte inesgotável de 
incentivo e inspiração para que eu perseguisse meus sonhos e me tornasse a pessoa que sou hoje. Levo 
comigo os valores e a sabedoria que eles compartilharam comigo durante sua vida.

Além disso, sou extremamente grato à Igreja Batista Regular do Catolé por me acolher e cuidar de mim 
durante todo esse tempo. A comunidade e o apoio espiritual que encontrei nesta igreja foram 
fundamentais para o meu desenvolvimento pessoal e acadêmico.

Por fim, gostaria de agradecer a todos os amigos, colegas e professores que, de alguma forma, 
contribuíram para a realização deste trabalho e estiveram ao meu lado durante essa jornada.

A todos vocês, meu mais sincero agradecimento.