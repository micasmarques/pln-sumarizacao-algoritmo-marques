\chapter{Considerações Finais e Trabalhos Futuros}
\label{chap:conclusoes-e-trabalhos-futuros}

Este trabalho, teve como objetivo comparar o desempenho de seis algoritmos de sumarização 
automática de texto em português, a saber: Algoritmo de Luhn, GistSumm, ChatGPT, Algoritmo PLI, Algoritmo de Regressão Bayesiana e Algoritmo de Marques. A metodologia utilizada consistiu na aplicação dos algoritmos em um texto sobre a COVID-19 e seus impactos na pandemia, e na avaliação dos resultados com base em métricas de qualidade de sumarização.

Os resultados obtidos indicaram que o algoritmo de Marques apresentou desempenho superior em relação aos 
outros algoritmos nas métricas de precisão, coerência, coesão e tempo de processamento. Esse resultado era 
esperado, visto que o algoritmo de Marques foi desenvolvido com base em técnicas avançadas de processamento de 
linguagem natural e aprendizado de máquina, e implementado com o objetivo específico de melhorar a qualidade da 
sumarização automática em português.

No que diz respeito às métricas avaliadas, o algoritmo de Marques apresentou valores de 0.8 para 
precisão, 0.9 para coesão e 0.75 para coerência, indicando que o resumo gerado por ele contém mais 
informações relevantes do texto original e é mais completo. Além disso, o algoritmo de Marques obteve 
a melhor taxa de coesão, indicando que ele conseguiu incluir as ideias mais importantes do texto 
original no resumo, enquanto mantinha uma taxa de coerência razoável. Em relação ao tempo de 
processamento, o algoritmo de Marques teve um desempenho satisfatório em comparação aos outros 
algoritmos.

Os resultados da comparação dos algoritmos de sumarização automática mostraram que o algoritmo proposto pelo 
autor do presente trabalho, o algoritmo de Marques, é uma escolha promissora para a geração de resumos automáticos 
de documentos em português. Ele apresentou desempenho superior em relação aos outros algoritmos testados e foi 
capaz de identificar as principais ideias presentes nos documentos analisados de forma precisa e completa.

No entanto, é importante ressaltar que a sumarização automática de texto ainda é uma área em 
desenvolvimento e que existem muitos desafios a serem superados para que os algoritmos de 
sumarização sejam capazes de produzir resumos realmente úteis e relevantes para os usuários. Alguns dos principais desafios incluem:

\begin{itemize}
    \item Adaptar os algoritmos para diferentes tipos de textos e domínios de conhecimento, garantindo que os resumos gerados sejam relevantes para os usuários em diferentes contextos;
    \item Avaliar os resumos gerados de forma mais rigorosa e consistente, considerando as expectativas e necessidades dos usuários;
    \item Lidar com a ambiguidade e a complexidade da linguagem natural, garantindo que os resumos mantenham a informação correta e essencial do texto original;
    \item Desenvolver técnicas de sumarização que levem em consideração as preferências e o conhecimento prévio dos usuários, gerando resumos personalizados de acordo com as necessidades individuais;
    \item Aprimorar o desempenho dos algoritmos em relação ao tempo de processamento e recursos computacionais, tornando-os mais eficientes e escaláveis.
\end{itemize}

Além disso, outro desafio importante é a necessidade de avaliar os resumos gerados de forma mais rigorosa e consistente, de forma a garantir que as métricas utilizadas para avaliar a qualidade da sumarização realmente refletem as necessidades e expectativas dos usuários. Nesse sentido, é importante que os algoritmos de sumarização sejam avaliados não apenas com base em métricas automatizadas, mas também, com base na avaliação humana, de forma a garantir que os resumos gerados sejam realmente úteis e relevantes para os usuários.

\section{Desafios e Avanços na Sumarização Automática de Texto}
\label{sec:desafios-e-avancos}

A sumarização automática de texto é uma área em constante desenvolvimento, com diversos desafios a serem superados 
e avanços a serem alcançados. Nesta seção, discutiremos os principais desafios enfrentados pela área e alguns dos 
avanços recentes que têm contribuído para a melhoria dos algoritmos de sumarização.

\subsection{Desafios}
Os principais desafios enfrentados na área de sumarização automática de texto incluem:

\begin{itemize}
    \item Adaptação a diferentes tipos de texto e domínios de conhecimento;
    \item Avaliação rigorosa e consistente dos resumos gerados;
    \item Lidar com ambiguidade e complexidade da linguagem natural;
    \item Geração de resumos personalizados;
    \item Melhoria de desempenho e escalabilidade.
\end{itemize}

\subsection{Avanços}
Alguns avanços recentes na área de sumarização automática de texto têm ajudado a enfrentar esses desafios, incluindo:

\begin{itemize}
    \item Abordagens baseadas em aprendizado profundo;
    \item Avaliação com base em métricas avançadas e análise humana;
    \item Uso de técnicas de transferência de aprendizado;
    \item Desenvolvimento de abordagens personalizadas;
    \item Otimização de recursos computacionais e tempo de processamento.
\end{itemize}

Em suma, a área de sumarização automática de texto tem enfrentado diversos desafios ao longo dos anos, mas também tem visto avanços significativos em termos de qualidade e eficiência dos resumos gerados. A pesquisa e o desenvolvimento contínuo de novas abordagens e técnicas são fundamentais para garantir que os algoritmos de sumarização sejam cada vez mais úteis e relevantes para os usuários, atendendo às suas necessidades e expectativas em diferentes contextos e situações.

Em conclusão, este trabalho contribuiu para o avanço da área de sumarização automática de texto em 
português, ao comparar o desempenho de seis algoritmos de sumarização e identificar o algoritmo 
de Marques como uma escolha promissora para a geração de resumos automáticos de documentos em 
português. No entanto, é importante destacar que a área ainda enfrenta desafios a serem superados, 
e é necessário continuar a pesquisa e o desenvolvimento de novos algoritmos e técnicas para 
garantir que os resumos gerados sejam cada vez mais úteis e relevantes para os usuários.

Dessa forma, sugere-se como trabalhos futuros o aprimoramento do algoritmo de Marques, visando a adaptação para diferentes tipos de textos e a avaliação humana mais rigorosa e consistente dos resumos gerados. Além disso, é necessário continuar a pesquisa em outras áreas relacionadas à sumarização automática de texto, como a identificação de informações relevantes e a geração de resumos personalizados de acordo com as necessidades dos usuários.