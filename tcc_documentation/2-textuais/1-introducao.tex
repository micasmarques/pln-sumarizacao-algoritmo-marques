\chapter{Introdução}
\label{cap:introducao}

Esse capítulo, apresenta a introdução da presente pesquisa. Na Seção \ref{sec:justificativa} são demonstradas a justificativa do trabalho, a importância para a área e o tema da pesquisa escolhida. Na Seção \ref{sec:objetivo}, são descritos os objetivos gerais e específicos. Por sua vez, a Seção \ref{sec:relevancia} discute a relevância da pesquisa. A Seção \ref{sec:contribuicoes} lista as principais contribuições que se deseja alcançar com o trabalho. Por fim, a Seção \ref{sec:descobertas-relevantes}, sumariza as principais descobertas realizadas no presente trabalho.

Diante da necessidade de usar o computador para ler e editar textos, foi preciso ``ensinar'' o computador a identificar textos em uma linguagem natural, e codificá-los para uma linguagem computacional. De acordo com \citeonline{chowdhary}, ocorreram vários avanços por meio de pesquisas na área de Inteligência Artificial (IA), o que possibilitou, a partir de uma de suas subáreas, uma forma para ensinar o computador a identificar palavras, chamada de Processamento de Linguagem Natural (PLN) \cite{nadkarni2011natural}.

O PLN \cite{nadkarni2011natural} busca soluções para questões computacionais, por meio de uma 
aprendizagem automática no processamento de linguagem e se dedica a propor e desenvolver modelos computacionais para a realização de tarefas que dependem da língua humana, escrita, como objeto primário. Para isso, linguistas e cientistas da computação, buscam fundamentos em várias disciplinas: Filosofia da Linguagem, Psicologia, Lógica, Inteligência Artificial, Matemática, Ciência da Computação, Linguística 
Computacional e Linguística cite{nadkarni2011natural}.

Segundo \citeonline{gariba}, o PLN
busca facilitar a interação do software com o usuário, o que possibilita além do
melhor entendimento, uma forma alternativa de alcançar o que se está procurando.

\section{Justificativa}
\label{sec:justificativa}

A sumarização automática de texto \cite{gonzalezlima} permite um contato acessível informacional ao leitor, sendo assim de 
fundamental importância \cite{leite2010estudo}. Para tal, este trabalho de conclusão de curso se dispõe a analisar técnicas de PLN,
e o uso de dados estatísticos específicos (por exemplo, frequência de  palavras-chave, número de ocorrências de entidades nomeadas, 
etc). O desafio consiste em uma redução da dimensionalidade e condensação semântica sem gerar prejuízo ao entendimento 
\cite{souza2017metodo}.

De acordo com \citeonline{balage2006estrutura, carlson2003building, lin2003automatic}, o processo de sumarização automática de textos adiantaria os estudos e trabalhos de diversos estudantes e pesquisadores. Uns para seus 
estudos, e outros para elaboração de artigos, dissertações, teses e pesquisas. 
Esse processo será explicado no decorrer desta pesquisa.

De acordo também com \citeonline{ribaldo2011sumarizaccao}, existe uma barreira
na sumarização automática que é identificar segmentos relevantes de um texto e 
compô-los, para produzir os sumários. Sumários, nesse caso, são os textos 
resumidos. O problema é importante, pois através dessa automatização, a 
sumarização transmitirá somente o que é importante de uma fonte textual de 
informação, tornando o resumo amplamente informativo, e com todas as 
informações que foram passadas no texto original 
\cite{black1988practical,torres2014automatic,barzilay1997michael}. Mas, também 
mostra que avaliar é um processo custoso, pois é imprescindível o uso do 
julgamento humano. Métodos avaliativos automáticos existem, mas não são tão 
satisfatórios quanto o julgamento humano, onde o mesmo julgamento humano pode 
ser problemático, dada a abstração desta tarefa \cite{eco2023definicao}.

\section{Objetivos}
\label{sec:objetivo}
Nesta seção, serão explicados os objetivos geral e específicos que esta pesquisa se propõe a alcançar.

\subsection{Objetivo Geral}
\label{sec:objgeral}
O objetivo geral deste trabalho é analisar e identificar os desafios e avanços em técnicas de 
processamento de linguagem natural na sumarização automática de textos. Para isso, será realizada uma análise comparativa entre diferentes algoritmos, buscando identificar aquele que apresenta melhor desempenho na geração de resumos de alta qualidade. O foco é identificar as dificuldades enfrentadas pela área e possíveis soluções para superá-las, a fim de gerar resumos que transmitam informações 
relevantes e essenciais do texto original, facilitando a compreensão do leitor.

\subsection{Objetivos Específicos}
\label{sec:objespecificos}

Para alcançar o objetivo geral, os seguintes objetivos específicos foram elencados:

\begin{itemize}
    \item Identificar os principais desafios da sumarização automática de texto;
    \item Analisar os algoritmos de sumarização automática existentes na literatura e compará-los de forma crítica;
    \item Propor um novo algoritmo de sumarização automática que leve em consideração os desafios identificados;
    \item Avaliar a qualidade dos resumos gerados pelo novo algoritmo em comparação com os algoritmos existentes;
    \item Contribuir para o avanço da área de sumarização automática de texto, apresentando resultados relevantes e conclusões embasadas na pesquisa realizada;
    \item Escrever e apresentar a pesquisa em formato de Monografia, seguindo as normas e diretrizes estabelecidas pela instituição de ensino.
\end{itemize}

\section{Relevância}
\label{sec:relevancia}

Ao atingir os objetivos citados na Seção \ref{sec:objetivo}, esta pesquisa conseguirá ajudar pesquisadores, acadêmicos, alunos e professores que precisam ler diversos textos em pouco tempo, e com a ajuda dessa pesquisa, conseguirão ter acesso a um resumo inteligente \cite{de2004avaliaccao}, facilitando assim o processo de leitura e escrita.

A presente pesquisa também ajudará estudantes da área de ciência da computação, pois pode ser um ponto de partida para aprender, na prática, como a Tecnologia da Informação (TI) está presente no dia a dia da população \cite{margaridosumarizaccao}.

Além disso, a pesquisa ajudará estudantes de outros cursos superiores que precisam escrever textos, artigos, pesquisas e monografias, e, ao se depararem com o processo de criação do resumo, não precisarão mais criar seu resumo tendo que ler tudo que escreveram, pois os algoritmos citados nessa monografia poderão ser usados para realizar essa tarefa.

\section{Contribuições}
\label{sec:contribuicoes}

Uma vez que não raros são os casos que se faz necessária a obtenção de resumos de determinados textos e o quanto pode ser um processo trabalhoso a realização dos mesmos, principalmente dependendo do tamanho, torna-se relevante um método que possibilite a sumarização automática \cite{barbieri2021sumarizaccao}. 

Para isso, existem alguns algoritmos de mineração de textos que proporcionam resumos automatizados. Destaca-se o clássico algoritmo de Luhn \cite{luhn1957stoical}, além dos algoritmos de \textit{GistSumm} \cite{muller2015comparativo}, o algoritmo de Programação Linear Inteira (PLI) \cite{oliveira2018sumarizaccao}, o algoritmo de regressão Bayesiana \cite{sodre2019avaliando} e o \textit{ChatGPT} \cite{rudolph2023chatgpt}.

Neste trabalho, o objetivo é realizar uma análise comparativa entre diferentes algoritmos de sumarização automática, buscando identificar aquele que apresenta melhor desempenho na geração de resumos de qualidade. Para tal, serão avaliados seis algoritmos de sumarização automática em um texto sobre a COVID-19 e seus impactos na pandemia, utilizando métricas de qualidade de sumarização. Além disso, serão discutidos os principais desafios enfrentados na área de sumarização automática de texto, contribuindo assim para o avanço do conhecimento e a superação de obstáculos nesse campo de pesquisa.


\section{Descobertas Relevantes do Trabalho}
\label{sec:descobertas-relevantes}

Neste trabalho de conclusão de curso, aborda-se o desafio da sumarização automática de texto, comparando o desempenho de seis algoritmos de sumarização. Os principais achados obtidos durante a execução deste trabalho são apresentados a seguir:
\begin{itemize}
    \item A aplicação e análise dos seis algoritmos de sumarização automática (Algoritmo de Luhn, \textit{GistSumm}, \textit{ChatGPT}, Algoritmo de Programação Linear Inteira, Algoritmo de Regressão Bayesiana e Algoritmo de Marques) em um texto sobre a COVID-19 e seus impactos na pandemia, revelou diferenças significativas no desempenho desses algoritmos em termos de precisão, coerência, coesão e tempo de processamento.
    \item Com base em métricas de qualidade de sumarização, o Algoritmo de Marques demonstrou desempenho superior em relação aos outros algoritmos, sendo identificado como uma escolha promissora para a geração de resumos automáticos de documentos. Este algoritmo apresentou resumos mais relevantes, coesos e coerentes, com menor tempo de processamento.
    \item Um estudo com usuários foi conduzido para verificar a utilidade e relevância dos resumos gerados pelos algoritmos de sumarização. Os resultados indicaram uma preferência geral pelo Algoritmo de Marques, confirmando os achados obtidos por meio das métricas de qualidade de sumarização.
    \item A pesquisa também destacou a necessidade de aprimorar os algoritmos de sumarização para lidar com diferentes tipos de textos e domínios de conhecimento, além de buscar a geração de resumos personalizados de acordo com as necessidades e expectativas dos usuários.
\end{itemize}

Em síntese, é esperado que o presente trabalho contribua para a área de sumarização automática de texto, dada a identificação do Algoritmo de Marques como uma opção eficaz para a geração de resumos automáticos. Os resultados obtidos reforçam a importância de continuar a pesquisa e desenvolvimento nesta área, buscando algoritmos e técnicas que gerem resumos cada vez mais úteis e relevantes para os usuários.