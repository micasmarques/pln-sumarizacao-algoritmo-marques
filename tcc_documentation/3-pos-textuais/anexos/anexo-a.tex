\anexo{Texto usado para gerar os Resumos e Resumos Gerados}
\label{an:forms}

\section{Texto da Covid 19 utilizado para fazer resumos}
\label{chap:texto_covid}
A pandemia da doença pelo coronavírus 2019, COVID-19 (sigla em inglês para coronavírus disease 2019) foi reconhecida pela Organização Mundial da Saúde (OMS) no dia 11 de março de 2020. No Brasil, desde o primeiro caso, confirmado em 26 de fevereiro, foram registrados outros 374.898, e 23.485 óbitos atestados até 1º de junho de 2020.

Uma importante questão epidemiológica diz respeito à elevada infectividade do SARS-CoV-2 (sigla em inglês para severe acute respiratory syndrome coronavirus 2), agente etiológico da COVID-19, cuja velocidade de propagação pode variar de 1,6 a 4,1. A elevada infectividade do SARS-CoV-2 e a ausência de uma vacina contra esse vírus fazem com que o aumento do número de casos seja exponencial.

Em função da inexistência de medidas preventivas ou terapêuticas específicas para a COVID-19, e sua rápida taxa de transmissão e contaminação, a OMS recomendou aos governos a adoção de intervenções não farmacológicas (INF), as quais incluem medidas de alcance individual (lavagem das mãos, uso de máscaras e restrição social), ambiental (limpeza rotineira de ambientes e superfícies) e comunitário (restrição ou proibição ao funcionamento de escolas e universidades, locais de convívio comunitário, transporte público,
além de outros espaços onde pode haver aglomeração de pessoas). Entre todas, destaca-se a restrição social.

No Brasil, diversas medidas foram adotadas pelos estados e municípios, como o fechamento de escolas e comércios não essenciais. Trabalhadores foram orientados a desenvolver suas atividades em casa, alguns municípios e estados encerraram-se em seus limites e divisas. Autoridades públicas locais chegaram a decretar bloqueio total (lockdown), com punições para estabelecimentos e indivíduos que não se adequassem às normativas. A restrição social resulta ser a medida mais difundida pelas autoridades, e a mais efetiva para evitar a disseminação da doença e achatar a curva de transmissão do coronavírus. Geralmente, a repercussão clínica e comportamental dessa obrigação implica mudanças no estilo de vida e pode afetar a saúde mental dos cidadãos.

Em relação aos estilos de vida, a restrição social pode levar a uma redução importante nos níveis de atividade física de intensidade moderada a vigorosa, e no aumento de tempo em comportamento sedentário. Nos Estados Unidos, observou-se um aumento no hábito de assitir à televisão (TV) e internet entre adultos durante a pandemia. Resultados semelhantes
foram identificados na Itália e na Espanha, tanto na participação em transmissões ao vivo, pelas redes sociais, quanto no aumento na instalação de aplicativos de programação de TV.

Outra preocupação refere-se à alteração dos hábitos alimentares. Nos Estados Unidos, no início da pandemia, observou-se um crescimento no volume de compras em supermercados e estoque doméstico de alimentos ultraprocessados e de alta densidade energética, como batatas fritas, pipoca, chocolate e sorvete. Adicionalmente, estudos indicam aumento no consumo de álcool, isoladamente, e no consumo associado de álcool e tabaco, durante a quarentena.

A adoção bem-sucedida de restrição social como medida de Saúde Pública traz comprovados benefícios à redução da taxa de transmissão da COVID-19; entretanto, efeitos negativos, associados a essa restrição, poderão ter consequências para a saúde, no médio e longo prazo. Portanto, espera-se das ações de Saúde Pública, também, uma capacidade de minimizar os efeitos adversos da restrição social prolongada.

\section{Resumo gerado pelo Algoritmo de Luhn}
\label{chap:luhn_resumo}
Em função da inexistência de medidas preventivas ou terapêuticas específicas para a COVID-19, e sua rápida taxa de transmissão e contaminação, a OMS recomendou aos governos a adoção de intervenções não farmacológicas (INF), as quais incluem medidas de alcance individual (lavagem das mãos, uso de máscaras e restrição social), ambiental (limpeza rotineira de ambientes e superfícies) e comunitário (restrição ou proibição ao funcionamento de escolas e universidades, locais de convívio comunitário, transporte público, além de outros espaços onde pode haver aglomeração de pessoas).
A restrição social resulta ser a medida mais difundida pelas autoridades, e a mais efetiva para evitar a disseminação da doença e achatar a curva de transmissão do coronavírus.
Em relação aos estilos de vida, a restrição social pode levar a uma redução importante nos níveis de atividade física de intensidade moderada a vigorosa, e no aumento de tempo em comportamento sedentário.
Nos Estados Unidos, observou-se um aumento no hábito de assitir à televisão (TV) e internet entre adultos durante a pandemia.
A adoção bem-sucedida de restrição social como medida de Saúde Pública traz comprovados benefícios à redução da taxa de transmissão da COVID-19; entretanto, efeitos negativos, associados a essa restrição, poderão ter consequências para a saúde, no médio e longo prazo.

\section{Resumo gerado pelo Algoritmo \textit{Gistsumm}}
\label{chap:gistsumm_resumo}
A pandemia da doença pelo coronavírus 2019, COVID-19 (sigla em inglês para coronavírus disease 2019) foi reconhecida pela Organização Mundial da Saúde (OMS) no dia 11 de março de 2020.
Uma importante questão epidemiológica diz respeito à elevada infectividade do SARS-CoV-2 (sigla em inglês para severe acute respiratory syndrome coronavirus 2), agente etiológico da COVID-19, cuja velocidade de propagação pode variar de 1,6 a 4,1.
Autoridades públicas locais chegaram a decretar bloqueio total (lockdown), com punições para estabelecimentos e indivíduos que não se adequassem às normativas.
Nos Estados Unidos, observou-se um aumento no hábito de assitir à televisão (TV) e internet entre adultos durante a pandemia.
A adoção bem-sucedida de restrição social como medida de Saúde Pública traz comprovados benefícios à redução da taxa de transmissão da COVID-19; entretanto, efeitos negativos, associados a essa restrição, poderão ter consequências para a saúde, no médio e longo prazo.

\section{Resumo gerado pelo Algoritmo Programação Linear Inteira}
\label{chap:pli_resumo}
A pandemia da doença pelo coronavírus 2019, COVID-19 (sigla em inglês para coronavírus disease 2019) foi reconhecida pela Organização Mundial da Saúde (OMS) no dia 11 de março de 2020.
Uma importante questão epidemiológica diz respeito à elevada infectividade do SARS-CoV-2 (sigla em inglês para severe acute respiratory syndrome coronavirus 2), agente etiológico da COVID-19, cuja velocidade de propagação pode variar de 1,6 a 4,1.
No Brasil, diversas medidas foram adotadas pelos estados e municípios, como o fechamento de escolas e comércios não essenciais.
Nos Estados Unidos, observou-se um aumento no hábito de assitir à televisão (TV) e internet entre adultos durante a pandemia.
A adoção bem-sucedida de restrição social como medida de Saúde Pública traz comprovados benefícios à redução da taxa de transmissão da COVID-19; entretanto, efeitos negativos, associados a essa restrição, poderão ter consequências para a saúde, no médio e longo prazo.

\section{Resumo gerado pelo Algoritmo de Regressão Bayesiana}
\label{chap:bayesiana_resumo}
Em função da inexistência de medidas preventivas ou terapêuticas específicas para a COVID-19, e sua rápida taxa de transmissão e contaminação, a OMS recomendou aos governos a adoção de intervenções não farmacológicas (INF), as quais incluem medidas de alcance individual (lavagem das mãos, uso de máscaras e restrição social), ambiental (limpeza rotineira de ambientes e superfícies) e comunitário (restrição ou proibição ao funcionamento de escolas e universidades, locais de convívio comunitário, transporte público,
além de outros espaços onde pode haver aglomeração de pessoas). A restrição social resulta ser a medida mais difundida pelas autoridades, e a mais efetiva para evitar a disseminação da doença e achatar a curva de transmissão do coronavírus. Em relação aos estilos de vida, a restrição social pode levar a uma redução importante nos níveis de atividade física de intensidade moderada a vigorosa, e no aumento de tempo em comportamento sedentário. Nos Estados Unidos, observou-se um aumento no hábito de assitir à televisão (TV) e internet entre adultos durante a pandemia. A adoção bem-sucedida de restrição social como medida de Saúde Pública traz comprovados benefícios à redução da taxa de transmissão da COVID-19; entretanto, efeitos negativos, associados a essa restrição, poderão ter consequências para a saúde, no médio e longo prazo.

\section{Resumo gerado pelo \textit{ChatGPT}}
\label{chap:chatgpt_resumo}
O texto discorre sobre a pandemia da COVID-19, destacando a elevada infectividade do SARS-CoV-2 e a ausência de uma vacina específica como fatores que contribuem para o aumento exponencial do número de casos. A Organização Mundial da Saúde (OMS) recomendou medidas preventivas não farmacológicas (INF), incluindo a restrição social, como forma de evitar a disseminação da doença. No Brasil, diversas medidas foram adotadas pelos estados e municípios, com destaque para a restrição social.
A restrição social pode afetar a saúde mental e os estilos de vida dos cidadãos, levando a uma redução nos níveis de atividade física e aumento do comportamento sedentário, bem como à alteração dos hábitos alimentares. Nos Estados Unidos, houve um aumento no hábito de assistir TV e internet e no consumo de alimentos ultraprocessados e de alta densidade energética, como batatas fritas, pipoca, chocolate e sorvete.
Embora a restrição social seja uma medida efetiva para reduzir a transmissão da COVID-19, pode ter consequências negativas para a saúde no médio e longo prazo. Portanto, é importante que as ações de Saúde Pública também considerem a minimização dos efeitos adversos da restrição social prolongada.


